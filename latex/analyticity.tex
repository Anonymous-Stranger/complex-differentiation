
We now consider how to systematically find where our expressions are analytic.
In general, this would require solving the Cauchy Riemann equations for
arbitrary functions, but the restrictions of our expression language allow us
to use a more tractable approach.


\begin{definition}[Countably invertible function]
  We say a function $f: \CC \to \CC$ is \emph{countably invertible} if the set
  of inputs to $f$ that attain a given value is countable, i.e.
  \[\forall w \in \CC \ldotp \exists m_f(w): \NN \twoheadrightarrow \{ z \mid f(z) = w \}. \]
  For our purposes, we also include complex infinity, $\infty \in \CC$, and
  refer to $m_f(w)$ as the counting function.
\end{definition}

Note first that our atoms, $z, \sin, \cos, \log, \cdots$ are countably
invertible, with the exception of a constant expression. We will show that
many combinations of expressions are also countably invertible, and use
this invertibility to find singularities.

\begin{lemma}
  \label{lem:ci-scalings}
  Scalings of countably invertible functions are countably invertible, i.e.
  if $f$ is countably invertible then so is $cf$, assuming
  $c \neq 0$.
\end{lemma}
\begin{proof}
  Because $f$ is countably invertible, it has a counting function $m_f(v)$.
  Then because $-f(w) = f(-w)$, we can construct $m_{-f}(w) = m_f(-w)$.
\end{proof}

\begin{lemma}
  \label{lem:ci-compositions}
  Compositions ($f \circ g$) of countably invertible functions are
  countably invertible.
\end{lemma}
\begin{proof}
  Let $f, g$ be countably invertible. We know there exist
  $m_f(u): \NN \twoheadrightarrow \{z \mid f(z) = u\}$, and
  $m_g(v): \NN \twoheadrightarrow \{z \mid g(z) = v\}.$
  Then if we fix a $w \in \CC$, we can define an onto mapping from $\NN^2$ to
  our inverses,
  \[m(w): \NN \times \NN \twoheadrightarrow \{z \mid f(g(z)) = w\},\]
  \[m(w, n_f, n_g) = m_g(m_f(w, n_f), n_g)).\]
  Then because $\NN^2$ is countably infinite, there is a bijection
  $p: \NN^2 \leftrightarrow \NN$, so we complete our proof by defining
  our counting function $m_{f \circ g}(w, n) = m(w, p(n)).$
\end{proof}

\begin{lemma}
  \label{lem:ci-inversions}
  Complex powers of countably invertible functions are countably invertible,
  i.e. if $f$ is countably invertible then $f^c$ is countably invertible
  for $c \neq 0$.
\end{lemma}
\begin{proof}
  Note first that $f(z)^c = e^{c\log(f(z))} = e^c e^{\log(f(z))}$.
  Then because $f(z)$ is countably invertible, from
  Lemma~\ref{lem:ci-compositions} we know that $p = e^{\log(f(z))}$ has a
  counting function $m_p(v)$. Thus we can construct a counting function
  $m_{f^c}(w) = m_p(w/e^c).$
\end{proof}

We now use countable invertibility to compute a result on the inverses of
of expressions. Recall that for each atom $f \in (z, \sin(z), \cos(z),
\cdots)$, the associated counting function $m_f(w, n)$ provides the
inverses for $f$. For example,
\[m_{e^z}(w, n) = \log(w) + i2n\pi \quad;\quad m_{cz}(w, n) = w/c. \]
Thus, by leaving $n$ as a free variable, we have an expression that
describes all inverses of the atom.

This can be generalized to any countably invertible expression.
\begin{theorem}
  \label{thm:expression-inversion}
  Inverses of countably invertible expressions can be described
  as expressions with a finite number of free integer variables.
\end{theorem}
\begin{proof}
  We proceed by strong induction over the nesting depth of expressions.

  For our base case, an expression $f$ with nesting depth 0 has no
  subexpressions, i.e. it is one of our atoms. Then as we noted
  previously, its counting function is the inverse described as
  an expression, and the expression has at most 1 free integer variables.

  Now consider an expression $f$ with nesting dept $n$. Then it has
  subexpressions $f_1, f_2, \cdots, f_k$, each of which can have
  nesting depth up to $n - 1$. From the proofs of our earlier lemmas,
  we know that we can construct a counting function for $f$ using only
  the counting functions of the subexpressions. Since $f_1, \cdots, f_k$
  have nesting depth at most $n - 1$. By our inductive assumption,
  we know that their counting functions $m_{f_1}, \cdots, m_{f_k}$ are
  expressions with finite free integer variables
  $(n_{1,1}, \cdots, n_{1,j_1}), (n_{2,1}, \cdots n_{2,j_2}),
  \cdots, (n_{k,1}, \cdots, n_{k, j_k})$ respectively.
  Thus $m$ is just an expression, and has $\sum_{i=1}^k j_i$
  free variables.
\end{proof}

This result means we can invert many complicated functions,
which then allows us to solve for singularities.

\paragraph{Isolated Singularities}
Recall that at an isolated singularity, an expression $f$ is either
complex infinity $(\infty)$ or undefined. In our system, complex infinity
is part of $\CC$, and our functions are always defined. Thus, $f$
has a singularity where it is $\infty$. By
Theorem~\ref{thm:expression-inversion}, we construct $f^{-1}$, and we
know there is a singularity at $f^{-1}(\infty).$

\paragraph{Branch Points}
Branch points in our expressions are introduced by $\log$ and non-integer
powers of expressions. Thus, we can search through our expression for
instances of $\log(f: Expr)$, or $(f: Expr)^c$.
By Theorem~\ref{thm:expression-inversion} we know $f$ has an inverse,
so we can compute $f^{-1}(0)$ and $f^{-1}(\infty)$ to produce expressions
describing the branch points of the individual $\log$ or power. Then by
listing all those expressions, we can describe all branch points of our
initial $Expr$.


\paragraph{Cluster Points}
Having found expressions for isolated singularities and
branch points, we may wonder whether there are cluster points among or created
by our singularities.

\begin{lemma}
  If the expressions $f_1, \cdots, f_k$ describe isolated singularities
  and branch points of an expression, then any cluster point that can be
  derived using all of $f_1, \cdots, f_k$ can also be derived by only using
  some particular $f_j$.
\end{lemma}
\begin{proof}
  Suppose there was a sequence of $\{z_n\}_{n=1}^\infty$ of points converging to
  a different point $z$ such that each $z_i$ was described by at least one
  $f_j$.
  Then $z$ would be a cluster point. Note that because $f_1, \cdots, f_k$ are a
  finite number of expressions, there must be at least one $f_j$ such that an
  infinite number of points in our sequence $\{z_n\}$ are described by $f_j$
  (otherwise the sequence would be finite). This means that there is some
  sequence $\{z_{jn}\}_{n=1}^\infty \rightarrow z$ such that each $z_{ji}$ is
  described by $f_j$.
\end{proof}

\begin{lemma}
  If the expression $f$ containing free variables $n_1, \cdots, n_k$ describes
  a cluster point, then at least one of the free variables is $\infty$ at the
  cluster point.
\end{lemma}
\begin{proof}
  Suppose $f$ described a cluster point $c$ where no $n_i$ was $\infty$.
  Since $f$ describes a cluster point, there must be some sequence of
  $\{z_j\}$ that converges to that cluster point, and all $z_j$ used
  $n_1, \cdots, n_k \leq M$. Then because $n_1, \cdots, n_k \in \NN$, we
  know that there can be no more than $kM$ elements in our infinite sequence.
  This is a contradiction, so there must be some $n_i$ that grows to $\infty$.
\end{proof}

Combining these two lemmas we can describe all cluster points of our expression
by taking each of our singularity expressions and setting each of their free
variables to $\infty$, one at a time.


\paragraph{What about non-countably invertible expressions?}
It is easy to produce non-countably invertible expressions by using sums
or products of expressions. We cannot use the above techniques on said
expressions, i.e we cannot invert them. However, if we treat these sums
or products as a single variable (effectively a u-substitution), then
we can still solve for the sum/product rather than solving for $z$. We then
leave that last bit of work to the user. As an optimization, for the specific
case of finding where a product $e_1 e_2 \cdots e_n$ is zero or infinity,
we can simply combine the results for each subexpression $e_i$.
