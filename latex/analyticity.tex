
We now consider how to systematically find where our expressions are analytic.
Note first that our atoms, $z, \sin, \cos, \log, \cdots$ all have a finite
number of isolated singularities and branch points.

\begin{definition}[Mostly analytic function]
  We say a function is \emph{mostly nonzero, mostly analytic (MNMA)} if it has
  a countable number of zeroes and singularities.
\end{definition}

By definition, our atoms are all MNMA, with the exceptions of the functions
$f(z) = 0, f(z) = \infty$. We now can show that every composition of the other
atoms is MNMA, by noting the following result:

\begin{lemma}[MNMA Compositions are MNMA]
  Compositions of mostly nonzero, mostly analytic functions are mostly
  nonzero, mostly analytic.
\end{lemma}
\begin{proof}
  We proceed by cases. Consider two mostly analytic functions
  $f, g: \CC \to \CC$, with singularities $S_f, S_g \subset \CC$ respectively.
  Also let surjections $m_f: \N \to S_f,\ m_g: \N \to S_g$ count each
  singularity set.
  \begin{description}
    \item[$f + g, f * z$ is mostly analytic.] Consider any point $z$ such that
    both $f$ and $g$ are analytic. Then at such a point, $f(z) + g(z)$ and
    $f(z)g(z)$ must be analytic. Therefore, the singularities of these
    compositions must be a subset of $S_f \cup S_g$.
  \end{description}
\end{proof}

\begin{itemize}
  \item Isolated Singularities -> Rootfinding
  \item Branch Points -> Rootfinding + Singularities
  \item Cluster Points -> Evaluation + Singularities + Branch Points
\end{itemize}
