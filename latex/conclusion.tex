We have shown that our method is not only able to find and detect isolated singular points and branch points, but also cluster points, which WolframAlpha doesn't do. Furthermore, our method solves certain cases of singularities that WolframAlpha doesn't, like $1/sin(1/sin(z))$. We aren't able to solve some cases that WolframAlpha can, like $1/f(z)$ where f is a polynomial, but this would be easily fixed with future work. We also don't classify isolated singular points into poles and essential singularities, but this could also be solved with future work. Our algorithm is easily extensible, and it is possible for us to achieve performance much better than WolframAlpha's singularity finding.

There are many improvements that can be made to our algorithm and its implementation. The main issue with our current algorithm is our inability to completely simplify expressions, because this leads to incorrect answers on inputs like $z/(z^2-z)$. These kinds of problems could be mostly solved by introducing factoring and expanding into our simplify function, which would allow us to cancel common terms, like the $z$ in the earlier expression. Another change we could make to our simplification algorithm would be to allow branching. Currently, the algorithm greedily applies whatever rules match the current expression, but it might simplify something that would remove the possibility of other simplification steps, which is suboptimal. Branching would allow us to try more possibilities and get more simplified expressions. We could also simply add more rules. The next problem in our algorithm is our inability to solve most equalities consisting of Sums and Terms. We would be able to solve this to some extent by figuring out a way to factor/expand some expressions to produce compositions of polynomials and other functions, which we could then attempt to solve. However, polynomials of degree 5 and above wouldn't be solvable without numerical techniques. Some expressions, like $z+sin(z)$, couldn't be simplified into polynomials, so we would also need numerical root-finding for these expressions. This still wouldn't work if an expression in the equality contained quantifiers. While solving the problem in general is impossible, even for WolframAlpha and Maple, these techniques would greatly increase the number of functions our algorithm would be able to provide answers for. The other big problems in our system are our inability to simplify subexpressions containing quantifiers, and the fact that we use floating point numbers instead of $\pi$ and $e$. Quantifier simplification is an extremely hard problem because quantifiers are elements of \ZZ. An example of this difficulty is $c_1n_1+c_2n_2$, where $Re(c_1)/Re(c_2)$ or $Im(c_1)/Im(c_2)$ is irrational, because this would simplify to some subset of \CC, and it would likely be difficult or impossible to represent this subset in terms of quantifiers. The problem of $\pi$ and $e$, is solvable by introducing new elements into the AST, and changing how constant expressions are evaluated, but this would take lots of time to implement correctly.
