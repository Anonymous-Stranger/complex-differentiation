A complex function is analytic on some region $R$ if it has a Taylor series that converges to it pointwise on that region. Functions are differentiable on a region $R$ if and only if they are analytic on that region. This means that to take the derivative of a function, you must first know if and where it is analytic. We don't allow expressions that contain functions that are analytic nowhere, like $\overline z$, $Re(z)$, and $Im(z)$, so we only need to find singular points of functions that are mostly analytic. Finding these singular points enables various useful things. For example, if you could classify singular points as isolated, branch, and cluster points, and you could calculate the residues at the isolated singular points, then it would be possible to use the Cauchy-Residue theorem and indented contours to calculate the integral of any closed loop in the complex plane on the allowed functions.

Finding the singular points of a symbolic function is difficult because it is difficult to calculate a Taylor series and test its convergence on a symbolic function. Even if that were possible, it would be difficult to find the singular points, the set of points on which a pointwise convergent Taylor series couldn't be found. It is possible to find the singularities of a function in different ways, using our knowledge of the singularities of simpler expressions, like our method described in Section~\ref{sec:analyticity}. Other computer algebra systems, such as Maple and WolframAlpha use similar methods to calculate singular points in the functions they are given. Maple and WolframAlpha can both find all isolated singular points and branch points of arbitrary functions, although WolframAlpha reports singularities in terms of the inputs to functions (for example, the singularities for $sin(1/z)$ are reported in the $1/z$-plane) and very often fails. Our method allows the user to find all isolated singular points, branch points, and cluster points of functions that we can simplify correctly. When we are unable to solve for the roots of certain equations, we report the answer similar to WolframAlpha. Unlike Maple and WolframAlpha, we are unable to classify our singular points into poles and essential singular points, and we are unable to perform additional computations such as calculating residues at singular points. We are also unable to determine if infinity is a singular point, or the type of singular point at infinity.
